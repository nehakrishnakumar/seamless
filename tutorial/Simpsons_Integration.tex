\sekshun{Simpson's Rule Integration}
\label{Simpsons_Integration}
\index{Simpson's rule integration}
\index{integration!Simpson's rule}

The Simpson's rule method approximates the function with a quadratic. The particular flavor that
we are going to use here requires that the interval $[a,b]$ is subdivided into an even number of
subintervals of width $h=(b-a)/N$. 
The general Simpson's rule is given by
\begin{equation}
  \int_a^b f(x) dx \approx \frac{h}{3} \left[ f(a) + 
  4\sum_{\substack{n=1\\n \text{odd}}}^{N-1} f(x_n) + 
  4\sum_{\substack{n=2\\n \text{even}}}^{N-2} f(x_n) + 
  f(b) \right] \label{eq:simpsons}
\end{equation}
where $x_n$ is still given by Equation~\ref{eq:xn-trapezoid} and 
$h$ is still given by Equation~\ref{eq:subinterval-width}.

For a function $f$ which has a fourth derivative, the maximum error $E$ for the
Simpson's rule method is given by the following equation:
\begin{equation}
  E \leq \frac{(b-a)^5}{180 N^4} f^{(4)}(\xi) \label{eq:simpsons-max-error}
\end{equation}
for some $\xi$ in $[a,b]$.

The expected error for the Simpson's rule method for all of the tests is expected  
to be very low, so we will use a value of 0.00001 for all of them.

  \begin{enumspec}
  \item\spec{14} Test \chpl{simpsonsIntegrationTest.chpl}: 
    integrates test functions using Simpson's method according to 
    Requirements~\ref{req@4.1} through~\ref{req@4.4}.\\
    \begin{tabular}{r r p{6cm}} \toprule
      \textbf{output} x 4  & \chpl{stdout: true}   & test passed \\ 
                           & \chpl{stdout: false}  & test failed \\ \midrule
      \textbf{modules loaded} & \multicolumn{2}{l}{\chpl{testFunctions}} \\
                              & \multicolumn{2}{l}{\chpl{simpsonsIntegration}} \\ \midrule
      \textbf{requirements met} & \multicolumn{2}{l}{\meetsreq{4.1,4.2,4.3,4.4}} \\ \bottomrule
  \end{tabular}
  \end{enumspec}

\begin{chapeltest}{simpsonsIntegrationTest.chpl}
  A test for \lstinline{simpsonsIntegration} using $f(x) = \{x^3, 1/x, x\}$.
  \begin{chapelpre}
  \end{chapelpre}
  \begin{chapel}
use simpsonsIntegration;
use testFunctions;

var exact:real;
var calculated:real;
var maxErr:real;

exact = 0.25;
maxErr = 0.00001;
calculated = simpsonsIntegration(
  a = 0.0, b = 1.0, N = 100, f = f1);
writeln((abs(calculated - exact) <= maxErr));

exact = 4.605170;
maxErr = 0.00001;
calculated = simpsonsIntegration(
  a = 1.0, b = 100.0, N = 1000, f = f2);
writeln((abs(calculated - exact) <= maxErr));

exact = 12500000;
maxErr = 0.00001;
calculated = simpsonsIntegration(
  a = 0.0, b = 5000.0, N = 5000000, f = f3);
writeln((abs(calculated - exact) <= maxErr));

exact = 18000000;
maxErr = 0.00001;
calculated = simpsonsIntegration(
  a = 0.0, b = 6000.0, N = 6000000, f = f3);
writeln((abs(calculated - exact) <= maxErr));
  \end{chapel}
  \begin{chapelpost}
  \end{chapelpost}
  \begin{chapeloutput}
true
true
true
true
  \end{chapeloutput}
\end{chapeltest}

\begin{enumspec}
\item\spec{15} Function \chpl{simpsonsIntegration}: 
  computes the definite integral of a function by Simpson's
  method of numerical integration.\\
  \begin{tabular}{r r p{10cm}} \toprule
    \textbf{arguments} & \chpl{a:real} & lower bound \\ 
                       & \chpl{b:real} & upper bound \\ 
                       & \chpl{N:int}  & number of subintervals (must be even)\\ 
                       & \chpl{f}      & function \\ \midrule
    \textbf{return}    & \chpl{:real}  & definite integral 
      computed by Simpson's method of numerical integration
      per Equations~\ref{eq:simpsons} and \ref{eq:subinterval-width} 
      and the expression for $x_n$ in \ref{eq:xn-trapezoid}\\
    \textbf{requirements met} & \multicolumn{2}{l}{\meetsreq{1.5,2,3}} \\ \bottomrule
  \end{tabular}
\end{enumspec}

\begin{TODO}
  Add check that N is even.
\end{TODO}

\begin{chapelsource}{simpsonsIntegration.chpl}
  \begin{chapel}
proc simpsonsIntegration(a: real(64), b: real(64), N: int(64), f): real{
  var h: real(64) = (b - a)/N; 
  var sum: real(64) = f(a) + f(b);
  var x_n: real(64);
  for n in 1..N-1 by 2 {
    x_n = a + n * h;
    sum = sum + 4.0 * f(x_n);
  }
  for n in 2..N-2 by 2 {
    x_n = a + n * h;
    sum = sum + 2.0 * f(x_n);
  }
  return (h/3.0) * sum;
}
  \end{chapel}
\end{chapelsource}

