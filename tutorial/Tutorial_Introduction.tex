\sekshun{Introduction}
\label{Tutorial_Introduction}

Several characteristics of legacy code present challenges to the scientific computing community. 
These characteristics are symptoms of the fact that, in general, the codes are written by individuals who are 
scentists or engineers first and computer programmers second. 
Often times, a code is very poorly documented, and the test suite amounts to a few exaample 
input decks and expected output decks. The vast majority of modules, procedures, and objects within a code
will likely not have unit tests associated with them. 
In addition to basic shortfalls in documentation, there is often a lack of context in the documentation that
is available.
Although adaptation patterns can be found in a sizable percentage of recently developed science codes, 
the traditional scientific software stack lacks adequate abstractions and tools necessary
to port it to a new execution environment or apply it to a problem set different than what was 
originally intended.\cite{kang}
While scientists are incentivized and indeed desire to produce reusable, maintainable code, these deficiencies 
are roadblocks to that end.\cite{petre} 

\begin{TODO}
  Add paragraph on overhead of good software engineering. Integration of multiple tools.
\end{TODO}

The purpose of the \seamless framework for science code development is to provide scientists and engineers
with some tools to help them produce code that is reusable and maintainable. 
In the \seamless framework, a test-driven development 
approach is combined with concepts from literate programming\cite{knuth} and 
best practices in software engineering. The resulting framework encourages development of very modular, 
well-documented code that is robust and reusable.  Currently, the framework supports 
the Chapel~(\url{http://chapel.cray.com}) programming language, but extension to other languages is 
straightforward.  This extensibility is possible because the framework is designed around the \latex
typesetting system. When following the \seamless approach, the developer is essentially writing a \latex
document containing the requirements, documentation, specification, test suite, and source code. Tests and 
source code are extracted and the tests are run by executing a script that is provided with the framework.

In addition to this introductory chapter, the remainder of the tutorial is an example \seamless 
specification for a simple software package
to compute the integral of a one-dimensional function using various numerical integration techniques. 
More specifically, the tutorial solves the Rosetta Code numerical integration 
task\cite{rosetta-code-numerical-integration} in Chapel. 
As you will
see, the ``off the shelf'' format of a \lstinline{seamless} specification supports colored text boxes to capture
various aspects of the specification such as \textit{Notes}, \textit{Rationales}, and \textit{Futures}. In addition,
the tutorial contains an additional type of text box not normally present in a specification to explain 
background information about the \lstinline{seamless} framework in context of the tutorial. Here is an example of one
of these special text boxes used only in the tutorial:
\begin{seamlessnote}
  Example text box used to provide background on the \lstinline{seamless} approach in context of the tutorial.
\end{seamlessnote}
But before we get to the example use of \seamless, some background on test-driven development, literate programming,
and software engineering practices is in order.

\section{Test-Driven Development}
Test-driven development, or TDD, is the notion that developers will improve both the design and
accuracy of their code by \textit{writing the test} for a particular feature \textit{before writing the 
code} that implements the feature according to the specification. In other words, the TDD process 
begins with writing an automated test for code that does not yet 
exist. After a test is written for a particular feature defined in the specification, the 
programmer then writes the implementing code to get the test to pass. This process is repeated until
all features in the specification are implemented.  
The idea is that by writing tests before code, rather than after, the tests will help guide
the design in small, incremental steps. Over time, this creates a well-factored and robust
codebase that is easier to modify.

\subsection{A Typical TDD Process}\label{tdd-classic}

A typical TDD process goes something like this:
\begin{enumerate}
\item Create a test. The test should be short and test for only one code feature. The test
  should run automatically.
\item Make sure the test fails. Verifying test failure before writing code helps to ensure
  that you are indeed testing the intended feature.
\item Write the simplest code possible to make the test pass. The code doesn't necessarily 
  need to be `good' yet.  Try not to look ahead too far. Write just enough code to clear 
  the current error.
\item Refactor the code. Once the test passes, improve the code through refactoring. 
  Clean up duplication. Optimize. Perhaps implement some parallelization. Consider creating new abstractions or
  objects.  Refactoring is a key part of design, so spend an appropriate amount of time on this step.
\item Retest. Run the tests again to make sure you haven't changed any behavior.
\item Retest and refactor some more. Do you get the point on refactoring yet?
\end{enumerate}

Repeat the above cycle until your code is complete. This will, in theory, ensure that your code is
always as simple as possible and completely covered by tests. 

\subsection{TDD Aids Design}
Test-driven development aids software design in several ways. Deciding which test to write next forces you
to think about what functionality you need in your code in a systematic way. Deciding how to test your code for
a given functionality forces you to think about how that functionality should be implemented, driving you to 
consider how the rest of the code will interface with it. After the test passes, you refactor and refactor again,
improving the design of the code in incremental ways. Since you have working unit tests for all of the functionality
that has been implemented prior to a given refactoring stage, you can feel confident that the improvements you make
through refactoring will not break another part of the code.

Continual alignment of source code with relevant tests as described above tends to ensure that the resulting 
software package is made up of small functions and objects that are loosely coupled and have minimal side effects.
TDD encourages a good code structure with low coupling (different parts of the code have minimal dependencies
on each other) and high cohesion (code that is in the same unit is all related). Code written without tests will
tend to have low cohesion and high coupling characteristics, making it much harder to cover with tests
written after development.

\subsection{Tests as Code Documentation}
A case can be made in some domains (\eg web development) that automated test suites 
provide an alternate means of documenting code---that
the tests are, in essence, a detailed specification of the code's behavior. This is somewhat true in
technical computing, but full documentation of scientific and engineering software requires 
more than just brief comments and example output. Surely, documentation for a function that computes 
the electron-electron repulsion integral in a quantum chemistry code must have some description
of the type of electronic wavefunction for which the code is valid! 

\section{Literate Programming}

A typical computer program consists of a text file 
containing program code. Strewn throughout will likely be scant plain text descriptions separated out by 
``comment delimiters'' that document various aspects of the code.
Since the actual code itself is presented in a such a way that supports the syntax, ordering, and structure 
that the programming language (and hence compiler) requires, the code comments will
be relatively disorganized and disjointed if you are reading them for documentation purposes. 
The way a code suite is organized in source is generally much different than the way thorough documentation is 
developed. The plain text nature of the comments also greatly limits their information value.

In literate programming the emphasis is reversed. Instead of writing \textit{a lot of} code that contains 
\textit{some} plain text documentation, 
the literate programmer writes \textit{thorough, well-organized, and content-rich} documentation that contains 
\textit{modular and efficient} code. 
The result is that the commentary is no longer hidden within a program surrounded by 
comment delimiters; instead, it is made the main focus. 
The ``program'' becomes primarily a document directed at humans, with the 
code interspersed within the documentation, separated out by ``code delimiters'' so that it can be extracted 
out and processed into source code by literate programming tools. The nature of literate programming is 
summarized pretty well in a quote from the online documentation for the FunnelWeb literate programming
preprocessor:

\begin{quote}
``The effect of this simple shift of emphasis can be so profound as to change one's whole approach to
programming. Under the literate programming paradigm, the central activity of programming becomes that of 
conveying meaning to other intelligent beings rather than merely convincing the computer to behave in a 
particular way. It is the difference between performing and exposing a magic trick.'' 
\begin{flushright}
-FunnelWeb Tutorial Manual\cite{funnelweb-what-is-literate-programming}
\end{flushright}
\end{quote}

\label{literate-program-requirements}
The following list of requirements can be used to define a ``literate program:''\cite{childs}
\begin{enumerate}
\item The high-level language code and the associated documentation come from the same 
set of source files.
\item The documentation and high-level language code for a given aspect of the program should be 
adjacent to each other when presented to the reader.
\item The literate program should be subdivided in a logical way.
\item The program should be presented in an order that is logical from the standpoint of documentation
rather than to conform to syntactic constraints of the underlying programming language(s).
\item The documentation should include notes on open issues and future areas for development.
\item Most importantly, the documentation should include a description of the problem and its solution. 
This should include all aids such as mathematics and graphics that enhance communication of the problem 
statement and the understanding of its challenge.
\item Cross references, indices, and different fonts for text, high-level language keywords, 
variable names, and literals should be reasonably automatic and obvious in the source and the documentation.
\item The program is written in small chunks that include the documentation, definitions, and code.
\end{enumerate}

The documentation portion may be any text that aids the understanding of the problem solved by the code 
(\eg description of the algorithm that is implemented).  The documentation is often significantly 
longer than the code itself. Ideally, the problem is described in a way that is agnostic of the language
in which the code is written.  For example, documentation for code that integrates a function $f(x)$ would
have discussion of discontinuities, various integration methods available (\eg trapezoidal, Simpson), 
domain of integration, \etc. 
In addition to basic shortfalls in documentation and testing in scientific codes, a recent 
study highlighted the widespread lack of basic context in available documentation.\cite{petre}
Literate programming solves this problem, ensuring that context is created while the program
is written.

\section{The \texttt{seamless} Framework}
Test-driven development and literate programming are certainly compatible.  In fact, they are 
complementary and their combined use is a rare actual example of ``the whole is greater
than the sum of its parts,'' especially in the context of developing science code.  
In one document, we can clearly outline the problem to be solved, develop a test for the 
code that we want, and document the code that solves the problem. As this is done in an incremental
manner, the scientist develops the code that solves the right problem in an efficient and robust manner.
As will be seen below, the process also supports several fundamental aspects of good software
engineering.

\begin{TODO}
  Add better overview of \seamless. See intro slides.
\end{TODO}

\subsection{Writing Good Requirements}\label{good-requirements}

The \seamless framework begins first with writing a high-level requirements specification.
Failing to write down the requirements is the single biggest unnecessary risk a developer
can take in a software project, resulting in greatly diminished productivity. 
For any non-trivial project (more than a few days of coding for one programmer), 
the lack of a thorough requirements specification will always result in more time spent and lower quality code.
Even for trivial examples, a short, informal specification will at least help to ensure
accuracy of the resulting code.

Depending on the size and complexity of the code, the requirements specification can have
multiple levels (or \textit{layers}). 
It clearly defines the problem that the program will solve. 
At some point in development of the requirements specification,
the developer should evaluate available algorithms and consider how data produced from the 
program will be used.
Even if a specification is written solely for the benefit of a lone developer, the act of 
writing the specification---describing how the program works in 
detail---will force the developer to begin thinking about design of the program.

\subsection{The Code Development Process Within the \texttt{seamless} Framework}\label{tdd-better}
Once a high-level requirements specification is in hand, the remainder of the \seamless process is
used to decompose the requirements, document the solution, write tests, and write source code:
\begin{enumerate}
\item Document a part of the problem and its solution.
  \begin{enumerate}
  \item Describe a small part of the problem to be solved. The description should include    
all aids such as mathematics and graphics that enhance communication of the problem 
statement and the understanding of its challenge. 
  \item Solve the problem, again using all aids at your disposal (\eg math, graphics).
  \item Include appropriate references to higher level requirement specifications.
  \end{enumerate}
\item Create a test. 
\begin{enumerate}
  \item The test should be as short as possible and test for one solution in your overall
  problem.\footnote{Note here that ``one thing in your code'' is replaced with ``one
  solution in your overall problem.'' This change emphasizes the literate programming emphasis
  on documenting the problem and solution before writing code. Writing the test is another
  form of documenting the solution.}
  \item The test should run automatically.
  \item Make sure the test fails. 
\end{enumerate}
\item Create the code.
  \begin{enumerate}
  \item Write the simplest code possible to pass the test. 
  \item After the test passes, refactor to improve the code. 
  \item Run the tests again to make sure the code still passes.
  \item Refactor and retest some more.
  \end{enumerate}
\end{enumerate}

Repeat the above cycle until your code is complete. This process will help to ensure that your code will 
have the following characteristics:
\begin{itemize}
  \item completely documented
  \item simple
  \item readable
  \item completely covered by tests
  \item robust
  \item accurate
  \item maintainable
  \item reusable
\end{itemize}

\section{Tutorial Structure}

The following files make up this tutorial for \seamless:
\begin{description}
  \item[Makefile] supports `pdflatex` and `latex` to compile the LaTeX package into the tutorial
    pdf; also has targets to make source code and tests from the LaTeX package and run the tests (usage of these
    targets is described in the tutorial)
  \item[seamless.cls] provides the `seamless` LaTeX document class which is a modification of the `book` class
  \item[seamless.sty] provides several environments and commands used in the seamless approach (usage is described
    in the tutorial)
  \item[Numerical\_Integration.tex] the main LaTeX document with includes for the remaining LaTeX files 
    (see Chapter~\ref{Organization}, Organization, for details on the structure of the tutorial)
  \item[chapel\_listing.tex] used with the LaTeX `listings` package to properly format Chapel code
  \item[references.bib] contains bibliography entries for use by `bibtex`
\end{description}

Adapting the associated \LaTeX\xspace files for a new software project is straightforward. 
Once you've adapted the structure of the \LaTeX\xspace package in the \lstinline{\tutorial} directory for your
purposes, and you've written a decent requirement specification, you're ready to begin the process described in
Section \ref{tdd-better}. 

\section{\texttt{seamless} Does Not Fully Implement Literate Programming}

The \seamless framework borrows from literate programming, but it is not a full implementation of its
ideas.  While the approach that I've described meets the intent of the requirements outlined in Section 
\ref{literate-program-requirements} above, it only fully implement one of the two main concepts of
literate programming.\cite{knuth}
The first concept, already described at length, is that code should have good documentation with all of the
supporting mathematics and graphics necessary to convey its function.

The other main concept of literate programming is that the best order to explain the parts of a program 
is not necessarily going to be the same order that the compiler needs to process the code. 
For example, you might have

\begin{verbatim}
proc readInAtoms(filename:string) {
  var infile = open(filename, iomode.r);
  var reader = infile.reader();

  // 55 lines of error handling code

  readNuclei(reader);
  readBasis(reader);

}
\end{verbatim}

When first describing the function of the above block of code, the developer wants to focus on a description of 
opening the file 
and reading in data, not discussing the error handling just because the computer language requires it to be in 
between the open and the read. You probably prefer to discuss the main logic first, returning to the error-handling 
part at some later point in the documentation, perhaps in a section of the documentation that covers error-handling
for the entire software package.
Also, for a collaborator that is reviewing code to understand and perhaps contribute to it, having all of that
error handling present in the first encounter with the code block is very distracting. It is an impediment to 
understanding the main purpose of the code.

When you program in a true literate programming system, you 
get to write code in any order that supports documentation of the code. The literate programming system 
comes with a utility program, 
usually called \lstinline{tangle}, which permutes the code into the right order so that you can compile or execute it.
This aspect of literate programming is not implemented in the \seamless framework. This is a conscious decision 
that was made in developing the framework in order to allow the 
TDD approach to drive the order in which test and source code is developed and presented to the
reviewer.  The compromise allows us to take advantage of many of the characteristics of literate programming while
executing a TDD process.

