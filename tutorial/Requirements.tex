\sekshun{Requirements}
\label{Requirements}
\index{requirements}

\begin{seamlessnote}
  As described in Section~\ref{tdd-better}, we must begin with good requirements. In the example shown below,
  we begin with a scope that is a brief description of the software package, summarizing the code's high-level
  capabilities. The functional requirements then spell out the specific requirements in sufficient detail that 
  every line of code can be traced back to a labeled item (\eg \textbf{R1.1}). The convention used below is that
  similar requirements are nested together, and only the most deeply nested items are numbered in a given
  chain of parent/child nestings. Each labeled item inherits the language of all of its higher level parents.
  For example, in the requirements list
  \begin{description}
    \item The code shall take inputs \chpl{a} and \chpl{b}
      \begin{description}
        \item and \chpl{c}
          \begin{description}
            \item[\textbf{R1.1}] and compute \chpl{a + b - c}
            \item[\textbf{R1.2}] and compute \chpl{a - b + c}
          \end{description}
        \item[\textbf{R2}] and compute \chpl{a * b}
      \end{description}
  \end{description}
  the Requirement~\textbf{R1.1} is "the code shall take inputs \chpl{a} and \chpl{b} and \chpl{c}
  and compute \chpl{a + b - c}; however, the Requirement~\textbf{R2} is "the code shall take 
  inputs \chpl{a} and \chpl{b} and compute \chpl{a * b}.

  To place a requirement label, use the command \lstinline!\req{x}!, where \texttt{x} is the desired 
  number (\eg \lstinline!\req{1.1}!).
\end{seamlessnote}


\section{Scope}
\label{Scope}
\index{scope}

The scope of this application is the numerical integration of arbitrary functions 
to solve the Rosetta Code numerical integration task\cite{rosetta-code-numerical-integration}
in Chapel. Solving the task requires development of functions to calculate the definite 
integral of a function ($f(x)$) using rectangular (left, right, and midpoint), trapezium, and Simpson's methods.

\section{Functional Requirements}
\label{Functional_Requirements}
\index{functional requirements}

\begin{description}
  \item The code shall have functions to calculate the definite integral of a function ($f(x)$).
  \item Available methods of integration shall include:
  \begin{description}
    \item[\req{1.1}] left rectangular
      \item[\req{1.2}] right rectangular
        \item[\req{1.3}] midpoint rectangular
    \item[\req{1.4}] trapezoid
    \item[\req{1.5}] Simpson's 
  \end{description}
  \item[\req{2}] The integration functions shall take in the upper and lower bounds ($a$ and $b$) and the number of 
approximations to make in that range ($N$). 
  \item[\req{3}] The integration functions shall return the value for the integral.
  \item The test suite shall demonstrate the code's capability by showing the results for the following cases:
  \begin{description}
    \item[\req{4.1}]
    $f(x) = x^3$, where $x$ is $[0,1]$, with 100 approximations. The exact result is 1/4, or 0.25.
    \item[\req{4.2}]
    $f(x) = 1/x$, where $x$ is $[1,100]$, with 1,000 approximations. The exact result is the natural log of 100, or about 4.605170.
    \item[\req{4.3}]
    $f(x) = x$, where $x$ is $[0,5000]$, with 5,000,000 approximations. The exact result is 12,500,000.
    \item[\req{4.4}]
    $f(x) = x$, where $x$ is $[0,6000]$, with 6,000,000 approximations. The exact result is 18,000,000.
  \end{description}
\end{description}
