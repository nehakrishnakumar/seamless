\sekshun{Notation}
\label{Notation}
\index{notation}

\section{Code Chunks}
\begin{TODO}
  Add a note about labeling requirements and specifications. Also, a note about how
  requirements traceability is accomplished.
\end{TODO}

Three types of code chunks that make up the software package are 
presented in this specification and delineated
with the appropriate keyword in italics: 
\textit{Source, Helper, or Test}.  The filename containing the code
chunk is given in parenthesis following the keyword. A brief description 
of the code chunk is also listed.
The actual code is represented with a fixed-width font where keywords are
bold and comments are italicized. An example helper code chunk is listed
below.

\textit{Helper (testFunctions.chpl)}. Provide the functions used
in the tests.
\begin{chapel}
proc f1(x:real):real {
  return x**3;
} 
proc f2(x:real):real {
  return 1/x;
} 
proc f3(x:real):real {
  return x;
} 
\end{chapel}

\section{Examples}

Examples of how to use the software are also provided and delineated with the
keyword \textit{Example} followed by a description of the use case. Here is
an example of an \textit{Example}:

\textit{Example}. The following line of code calls the function
\chpl{leftRectangleIntegration} to perform the integral of the function \chpl{f1}
over the interval $[1.0,4.0]$ using the left rectangle method with 100 subdivisions
and stores the result in the variable \chpl{result}.
\begin{chapel}
var result: real = leftRectangleIntegration(
  a = 1.0, b = 4.0, N = 100, f = f1);
\end{chapel}

\section{Text Boxes}
Different color text boxes are used to highlight \textit{TODO's, Notes,
Rationales, Open Issues,} and \textit{Futures} throughout the specification.
Examples of these text boxes along with definitions for each of these terms
is given below:

\begin{TODO}
  Things that need to be done for this version of the software.
\end{TODO}

\begin{note}
  Something of note that does not fit into any other category.
\end{note}

\begin{rationale}
  An explanation for a particular design choice.
\end{rationale}

\begin{openissue}
  Issue that we do not know how to handle.
\end{openissue}

\begin{future}
  Issue or feature that we have a story about, but which is not yet
  fully-designed or implemented. 
\end{future}
