\sekshun{Organization}
\label{Organization}
\index{organization}

This tutorial is organized as follows:

\begin{description}

  \item[Chapter~\ref{Tutorial_Introduction}] Tutorial Introduction, background on 
    the test-driven development process of the \lstinline{seamless} package, 
    the importance of starting with good requirements, and a process to establish
    traceability of source code to requirements.

  \item[Chapter~\ref{Notation}] Notation, Introduces the notation that is used
    throughout a typical \lstinline{seamless} specification.

    %\item[Chapter~\ref{Acknowledgments}] Acknowledgements, offers a note of
    %thanks to people and projects.

  \item[Chapter~\ref{Organization}] Organization, describes the contents of
    each of the chapters within this tutorial.

  \item[Chapter~\ref{Requirements}] Requirements, scope and functional requirements for 
    the numerical integration code that we will develop in the tutorial.

  \item[Chapter~\ref{Rectangle_Integration}] Rectangle Integration, documentation, source
    code, and test suite for implementation of the rectangle method of numerical integration 
    in the Chapel language.

  \item[Chapter~\ref{Trapezoid_Integration}] Trapezoid Integration, documentation, source
    code, and test suite for implementation of the trapezoid method of numerical integration 
    in the Chapel language.

  \item[Chapter~\ref{Simpsons_Integration}] Simpson's Integration, documentation, source
    code, and test suite for implementation of the Simpson's method of numerical integration 
    in the Chapel language.

\end{description}
