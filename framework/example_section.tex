\sekshun{Trapezoid Integration}
\label{Trapezoid_Integration}
\index{trapezoid integration}
\index{integration!trapezoid}

The trapezoid method computes an approximation to a 
definite integral by finding the area of a collection of trapezoids whose heights are determined 
by the values of the function.  Specifically, the interval $[a,b]$ over which the function is to 
be integrated is divided into $N$ equal subintervals of length $h = (b-a)/N$. The trapezoids are 
drawn with the base along the $x$-axis.  Both the left and right corner of the side opposite the 
base lies on the graph of the function. The approximation to the integral is 
then calculated by adding up the areas of the trapezoids (base multiplied by sum of two sides 
divided by two) of the $N$ trapezoids, giving the formula:
\begin{equation}
  \int_a^b f(x) dx \approx \frac{h}{2} \left[ f(a) + 2\sum_{n=1}^{N-1} f(x_n) + f(b) \right] \label{eq:trapezoid}
\end{equation}
where 
\begin{equation}\label{eq:xn-trapezoid}
x_n = a + nh
\end{equation}
and $h$ is still given by Equation~\ref{eq:subinterval-width}.

For a function $f$ which is twice differentiable, the maximum error $E$ for the
trapezoid method is given by the following equation:
\begin{equation}
  E \leq \frac{(b-a)^3}{12 N^2} f''(\xi) \label{eq:trapezoid-max-error}
\end{equation}
for some $\xi$ in $[a,b]$.

We can use the maximum value of the second derivative 
computed in Section~\ref{sec:midpoint-rectangle-method} 
in Equation~\ref{eq:trapezoid-max-error}. 
As with the midpoint rectangle method, the trapezoid method is expected to give a very 
accurate answer for $f(x) = x$,
so we will use a value of 0.00001 for the maximum expected error for the two final tests.
The calculated maximum expected error for the tests specified in Requirements~\ref{req@5.1} and
\ref{req@5.2} are given in Table~\ref{tab:trapezoid-error}.

\begin{table}[htbp]
  \centering
  \caption{Values for expressions in Equation~\ref{eq:trapezoid-max-error} and the maximum 
    expected error of the trapezoid method of numerical integration for $f(x) = \{x^3, 1/x\}$.}
  \label{tab:trapezoid-error}
  \begin{tabular}{ccccc}
    \textbf{Function} & \textbf{Interval} & \textbf{N} & \textbf{Maximum $f''(x)$} & $E$  \\ \toprule
    $x^3$ & $[0,1]$   & 100  & 6 & 0.00005 \\ \midrule
    $1/x$ & $[1,100]$ & 1000 & 3 & 0.24257 \\ \bottomrule
  \end{tabular}
\end{table}

  \begin{enumspec}
  \item\spec{12} Test \chpl{trapezoidIntegrationTest}: 
    integrates test functions according to Requirements~\ref{req@4.1} through~\ref{req@4.4}
    using the trapezoid numerical integration method and compares results
    to the maximum expected error.\\
    \begin{tabular}{r r p{6cm}} \toprule
      \textbf{output} x 4  & \chpl{stdout: true}   & test passed \\ 
                           & \chpl{stdout: false}  & test failed \\ \midrule
      \textbf{modules loaded} & \multicolumn{2}{l}{\chpl{testFunctions}} \\
                              & \multicolumn{2}{l}{\chpl{trapezoidIntegration}} \\ \midrule
      \textbf{requirements met} & \multicolumn{2}{l}{\meetsreq{4.1,4.2,4.3,4.4}} \\ \bottomrule
  \end{tabular}
  \end{enumspec}

\begin{chapelexample}{trapezoidIntegrationTest.chpl}
  A test for \lstinline{trapezoidIntegration} using $f(x) = \{x^3, 1/x, x\}$.
  \begin{chapelpre}
  \end{chapelpre}
  \begin{chapel}
use trapezoidIntegration;
use testFunctions;

var exact:real;
var calculated:real;
var maxErr:real;

exact = 0.25;
maxErr = 0.00005;
calculated = trapezoidIntegration(
  a = 0.0, b = 1.0, N = 100, f = f1);
writeln((abs(calculated - exact) <= maxErr));

exact = 4.605170;
maxErr = 0.24257;
calculated = trapezoidIntegration(
  a = 1.0, b = 100.0, N = 1000, f = f2);
writeln((abs(calculated - exact) <= maxErr));

exact = 12500000;
maxErr = 0.00001;
calculated = trapezoidIntegration(
  a = 0.0, b = 5000.0, N = 5000000, f = f3);
writeln((abs(calculated - exact) <= maxErr));

exact = 18000000;
maxErr = 0.00001;
calculated = trapezoidIntegration(
  a = 0.0, b = 6000.0, N = 6000000, f = f3);
writeln((abs(calculated - exact) <= maxErr));
  \end{chapel}
  \begin{chapelpost}
  \end{chapelpost}
  \begin{chapeloutput}
true
true
true
true
  \end{chapeloutput}
\end{chapelexample}

\begin{enumspec}
\item\spec{13} Function \chpl{trapezoidIntegration}: 
  computes the definite integral of a function by the trapezoid
  method of numerical integration.\\
  \begin{tabular}{r r p{10cm}} \toprule
    \textbf{arguments} & \chpl{a:real} & lower bound \\ 
                       & \chpl{b:real} & upper bound \\ 
                       & \chpl{N:int}  & number of subintervals \\ 
                       & \chpl{f}      & function \\ \midrule
    \textbf{return}    & \chpl{:real}  & definite integral 
      computed by trapezoid method of numerical integration
      per Equations~\ref{eq:trapezoid} and \ref{eq:subinterval-width} 
      and the expression for $x_n$ in \ref{eq:xn-trapezoid}\\
    \textbf{requirements met} & \multicolumn{2}{l}{\meetsreq{1.4,2,3}} \\ \bottomrule
  \end{tabular}
\end{enumspec}

\begin{chapelsource}{trapezoidIntegration.chpl}
  \begin{chapel}
    proc trapezoidIntegration(a: real(64), b: real(64), N: int(64), f): real{
      var h: real(64) = (b - a)/N; 
      var sum: real(64) = f(a) + f(b);
      var x_n: real(64);
      for n in 1..N-1 {
        x_n = a + n * h;
        sum = sum + 2.0 * f(x_n);
      }
      return (h/2.0) * sum;
    }
  \end{chapel}
\end{chapelsource}

