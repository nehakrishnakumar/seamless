\sekshun{Organization}
\label{Organization}
\index{organization}

This specification is organized as follows:

\begin{description}
\item[Part I] Introduction

\begin{description}

\item[Chapter~\ref{Notation}] Notation, introduces the notation that is used
throughout this document.

%\item[Chapter~\ref{Acknowledgments}] Acknowledgements, offers a note of
%thanks to people and projects.

\item[Chapter~\ref{Organization}] Organization, describes the contents of
each of the parts and chapters within this document.

\item[Chapter~\ref{Development_Approach}] Development Approach, describes 
the test-driven development process within the literate programming approach.
\end{description}

\item[Part II] Requirements Specification

\begin{description}
\item[Chapter~\ref{Scope}] Scope, describes the scope of the package.
\end{description}

\begin{description}
\item[Chapter~\ref{Functional_Requirements}] Functional Requirements, describes the 
functional requirements of the package.
\end{description}

\item[Part III] Technical Specification

\begin{description}
\item[Chapter~\ref{Numerical_Integration}] Numerical Integration, documentation, source
code, and test suite for implementation of numerical integration in the Chapel language.
\end{description}

%\item[Appendix~\ref{Syntax}] Collected Lexical and Syntax Productions,
%contains the syntax productions listed throughout this specification
%in both alphabetical and depth-first order.

\end{description}
