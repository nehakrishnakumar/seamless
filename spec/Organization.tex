\sekshun{Organization}
\label{Organization}
\index{organization}

This book is organized as follows:

\begin{description}

\item[Chapter~\ref{Notation}] Notation, introduces the notation that is used
throughout the book.

%\item[Chapter~\ref{Acknowledgments}] Acknowledgements, offers a note of
%thanks to people and projects.

\item[Chapter~\ref{Organization}] Organization, describes the contents of
each of the parts and chapters within this document.

\item[Chapter~\ref{Development_Approach}] Development Approach, describes 
the unique test-driven development process of the \lstinline{seamless} package.

\item[Chapter~\ref{Requirements}] Requirements, explains the importance of starting
with good requirements along with example scope and functional requirements for 
the numerical integration code that we will develop in the book.

\item[Chapter~\ref{Numerical_Integration}] Numerical Integration, documentation, source
code, and test suite for implementation of numerical integration in the Chapel language.

%\item[Appendix~\ref{Syntax}] Collected Lexical and Syntax Productions,
%contains the syntax productions listed throughout this specification
%in both alphabetical and depth-first order.

\end{description}
