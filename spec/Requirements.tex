\sekshun{Requirements}
\label{Requirements}
\index{requirements}

\begin{TODO}
Describe why you must begin with good requirements.
\end{TODO}

\section{Scope}
\label{Scope}
\index{scope}

The scope of this application is the numerical integration of arbitrary functions 
to solve the Rosetta Code numerical integration task\cite{rosetta-code-numerical-integration}
in Chapel. Solving the task requires development of functions to calculate the definite 
integral of a function ($f(x)$) using rectangular (left, right, and midpoint), trapezium, and Simpson's methods.

\section{Functional Requirements}
\label{Functional_Requirements}
\index{functional requirements}

\begin{future}
Consider creating a \LaTeX\xspace package to roll together the features in \lstinline{spec.tex}
for creating/cross-referencing lists of requirements and specifications and creating the 
requirement traceability matrix.
\end{future}

\begin{description}
  \item[\req{1}] The code shall have functions to calculate the definite integral of a function ($f(x)$).
  \item[\req{2}] Available methods of integration shall include:
  \begin{description}
    \item[\req{2.1}] rectangular
      \begin{description}
        \item[\req{2.1.1}] left
        \item[\req{2.1.2}] right
        \item[\req{2.1.3}] midpoint
      \end{description}
    \item[\req{2.2}] trapezium
    \item[\req{2.3}] Simpson's 
  \end{description}
  \item[\req{3}] The integration functions shall take in the upper and lower bounds ($a$ and $b$) and the number of 
approximations to make in that range ($N$). 
  \item[\req{4}] The integration functions shall return the value for the integral.
  \item[\req{5}] The test suite shall demonstrate the code's capability by showing the results for the following cases:
  \begin{description}
    \item[\req{5.1}]
    $f(x) = x^3$, where $x$ is $[0,1]$, with 100 approximations. The exact result is 1/4, or 0.25.
    \item[\req{5.2}]
    $f(x) = 1/x$, where $x$ is $[1,100]$, with 1,000 approximations. The exact result is the natural log of 100, or about 4.605170.
    \item[\req{5.3}]
    $f(x) = x$, where $x$ is $[0,5000]$, with 5,000,000 approximations. The exact result is 12,500,000.
    \item[\req{5.4}]
    $f(x) = x$, where $x$ is $[0,6000]$, with 6,000,000 approximations. The exact result is 18,000,000.
  \end{description}
\end{description}
